\documentclass[a4paper, 15pt,twocolumn]{article}
\usepackage{algorithm}
\usepackage{algpseudocode}
\usepackage{amsmath}

\title{Comparison Based Sorting Algorithms}
\author{Ankush Singh}
\date{}
\begin{document}
\maketitle
\begin{abstract}
This document presents a brief discussion on sorting algorithms. Algorithms for \texttt{Quicksort} is provided in this document and its working is explained.
Further, a proof of lower bounds on sorting is presented in this document.Most of the content presented here is created by referring and reproducing
contents from one of the widely followed book on
Algorithms by Cormen et al\cite{cor}.\textbf{We do not claim
originality of this work.} This document is 
prepared as part of an assignment for the Software Lab
Course (CS251) to learn \LaTeX.
\noindent
\fbox{\begin{minipage}{21em}
Declaration: I have prepared this document us-\newline
ing \LaTeX without any unfair means. Further,
while the document is prepared by me, I do not
claim the ownership of the ideas presented in
this document.
\end{minipage}}
\end{abstract}
\section{Introduction}

Sorting is one of the most fundamental operations
in computer science useful in numerous applications. Given a sequence of numbers as input, the output should provide a non-decreasing sequence of numbers as output. More formally, we define
sorting problem as follows [1],\\
\textbf{Input}: A sequence of $n$ numbers $\langle a_1 , a_2 , ..., a_n\rangle$.\\
\textbf{Output}: A reordered sequence $\langle$of size n$\rangle$
$\langle a_1^{'} , a_2^{'} , ..., a_n^{'}\rangle$ of the input sequence such that $a_1^{'}\leq a_2^{'} \leq ... \leq a_n^{'} $.\\
Consider the following example. Given an input
sequence $\langle8, 34, 7, 9, 15, 91, 15 \rangle$, a sorting algorithm
should return $\langle7, 8, 9, 15, 15, 34, 91\rangle$ as output.\\
A fundamental problem like sorting has attracted
many researchers who contributed with innovative
algorithms to solve the problem of sorting \textit{effi-
ciently} [3]. Efficiency of an algorithm depends on
primarily on two aspects,

\begin{itemize}
\item \textbf{Time complexity} is a formalism that cap-
tures running time of an algorithm in terms of the input size. Normally, \textit{asymptotic} behavior on the input size is used to analyze the time complexity of algorithms.
\item \textbf{Space Complexity} is a formalism that captures amount of mempry used by an algorithm in terms of input size. Like time complexity analysis, asymptotic analysis is used for space complexity.

\end{itemize}
In the branch of algorithms and complexity in com-
puter science, space complexity takes a back seat
compared to time complexity. Recently, another
parameter of computing i.e., energy consumption
has become popular. Roy et al. [4] proposed an energy complexity model for algorithms. In this document, we will deal with time complexity of sorting
algorithms.\\
One class of algorithms which are based on \textit{element comparison} are commonly known as \textit{comparison based sorting algorithms.} In this document we
will provide a brief overview of \texttt{Quicksort}, a commonly
used comparison based sorting algorithm [2].
Quicksort is a sorting algorithm based on\textit{ divide-
and-conquer} paradigm of algorithm design. Fur-
ther, we will derive the lower bound of any com-
parison based sorting algorithm to be $\Omega(n\log_2n)$
for an input size of n.


\section{Quicksort}
Quicksort is designed as a three-step divide-and-
conquer process for sorting an input sequence in
an array [1]. For any given subarray, $A[i..j]$, the
process is as follows,\\
\textbf{Divide}: The array $A[i..j]$ is partitioned into two
subarrays $A[i..k]$ and $A[k + 1..j]$ such that all ele-
ments in $A[i..k]$ is less than or equal to all elements
in $A[k + 1..j]$. A partitioning procedure is called to
determine k such that at the end of partitioning,
the element at the k th position (i.e., $A[k]$) does not
change its position in the final output array.
\\
\begin{algorithm}
  \caption{Partition procedure of \texttt{Quicksort} algorithm}
  \label{algo:ins_sort1}
  \begin{algorithmic}[1]
     \Procedure{Partition}{\textit{A},i,j}\newline
     \Comment{$A$ is an array of $N$ integers, $A[1...N]$}\newline
     \Comment{$i$ and $j$ are the start and end of subarray}
      \State $x \leftarrow A[i]$ 
      \State $y\leftarrow i-1$ 
      \State $z \leftarrow j+1$ 
      \While {$(true)$}
          \State $z \leftarrow z-1$ 
         \While {$A[z] > x$}
            \State $z\leftarrow z-1$
         \EndWhile
      \State $y \leftarrow y + 1$
      	 \While {$A[y]<x $}
            \State $y \leftarrow y + 1$
      	\EndWhile
         \If {$y < z$}
             \State Swap $\ A[y]\leftrightarrow A[z]$
         \Else 
      	     \State $return\ z$
         \EndIf
     \EndWhile
    \EndProcedure 
  \end{algorithmic}
\end{algorithm}
\textbf{Conquer}:Recursively invoke \texttt{Quicksort} on the two subarrays. This procedure conquers the complexity by applying the same operations in two sub-arrays.\\
\textbf{Merge}: Quicksort does not require merge or combine operation as the entire array $A[i..j]$ is sorted
in place.\\
In the heart of \texttt{Quicksort}, there is a partition
procedure as shown in Algorithm 1. A pivot element $x$ is selected. The first inner while loop (line
$\#$6) continues examining elements until it finds an
element that is smaller than or equal to the pivot element. Similarly, the second inner while loop (line
$\#$9) continues examining elements until it finds an
element that is greater than or equal to the pivot
element. If indices $y$ and $z$ have not exchanged
their side around the pivot, the elements at $A[y]$
and $A[z]$ are exchanged. Otherwise, the procedure
returns the index $z$, such that all elements to the
left of $z$ are smaller than or equal to $A[z]$ and all
elements to the right of $z$ are greater than or equal
to $A[z]$.\\
The main recursive procedure for \texttt{Quicksort} is
presented in Algorithm 2. Initial invocation is performed by call $QUICKSORT(A, 1, N )$ where $N$ is
the length of array N.
\begin{algorithm}
  \caption{\texttt{Quicksort} recursion}
  \label{algo:qui_sort1}
  \begin{algorithmic}[1]
     \Procedure{Quicksort}{\textit{A},$i$,$j$}\newline
     \Comment{Quicksort procedure called with $A,1,N$}
         \If {$i < j$}
             \State $k \ \leftarrow \textsc{Partition}(A,i,j)$
             \State \textsc{Quicksort}$(A,i,k)$
             \State \textsc{Quicksort}$(A,k+1,j)$
         \EndIf
     \EndProcedure 
  \end{algorithmic}
\end{algorithm}

\subsection{Time Complexity analysis of \texttt{Quicksort}}
The worst case of \texttt{Quicksort} occurs when an array of length N , gets partitioned into two subarrays
of size N-1 and 1 in every recursive invocation of
QUICKSORT procedure in Algorithm 2. The partitioning procedure presented in Algorithm 1, takes
$\Theta(n)$ time, the recurrence relation for running time
is,\\
\begin{equation*}
\textit{T}(n)=\textit{T}(n-1)+\Theta(n)
\end{equation*}
As it is evident that $\textit{T}(1)=\Theta(1)$, the recurrence is solved as follows:
\indent
\begin{align*}
\textit{T}(n)&= \textit{T}(n-1)+\Theta(n)\\
&= \sum_{k=1}^{n}\Theta(k)\\
&= \Theta\left(\sum_{k=1}^{n}k\right)	\\
&= \Theta(n^2)\\
\end{align*}


Therefore, if the partitioning is always maximally
unbalanced, the running time is $\Theta(n^2)$. Intutively,
if an input sequence is almost sorted, \texttt{Quicksort}
will perform poorly. In the best case, partitioning
divides the array into two equal parts. Thus, the
recurrence for the best case is given by,\\
\begin{equation*}
\textit{T}(n)=2\textit{T}\left(\frac{n}{2}\right)+\Theta(n)
\end{equation*}
which evaluates to $\Theta(n\log_2(n)$. Using a comparatively involved analysis, the average running time
of \texttt{Quicksort} can be determined to be $O(n\ lgn)$.
\section{Lower Bounds on comparison sorts}
An interesting question about sorting algorithms
based on comparisons is the following: What is
the lower bound of this class of sorting algorithms? This question is important for algorithm researchers to further improve the sorting algorithms.\\
A decision tree based analysis leads to the following \vspace{0.5pc} theorem [1].\\
\textbf{Theorem 1.} Any decision tree that sorts n elements has height \vspace{0.5pc}$\Omega(n\log_2(n)$.\\
\textit{Proof.} Consider a decision tree of height h that 
sorts n elements. Since there are n! permutations
of n elements, each permutation representing a distinct sorted order, the tree must have at least n! 
leaves. Since a binary tree of height h has no more
than $2^h$ leaves. So,\\
\vspace{0.5pc}$n!\leq 2^h$\\
Applying logarithmic ($\log_2$), the inequality becomes,\\
$h\geq lg(n!)$.\newline
Applyting Stirling's approximations,\\
$n!>\left(\frac{n}{e}\right)^n$ ,\\
where $e$ is natural base of algorithms. Further,
\indent
\begin{align*}
h&\geq\ lg\left(\frac{n}{e}\right)^n\\
&= n\ lgn -n\ lge\\
&= \Omega(n\ lgn)
\end{align*}
\section{Conclusion}
In this document, we have provided a discussion
on sorting algorithms. We included algorithms for
\texttt{Quicksort} and explained its working. Further, a
proof of lower bounds on sorting is presented in this
document. Most of the content presented here is
created by referring and reproducing contents from
one of the widely followed book on Algorithms by
Cormen et al. [1]. We do not claim originality of
this work. This document is prepared as part of an
assignment for the Software Lab Course (CS251) to
learn \LaTeX.



\begin{thebibliography}{9}
\bibitem{cor} 
\textsc{Cormen, T. H., Leiserson, C. E., Rivest, R. L., and Stein, C.}
\textit{Introduction to Algorithms, Third Edition}. 
3rd ed. The MIT Press, 2009.
 
\bibitem{hoare} 
\textsc{Hoare, C. A. R.} Algorithm 64: Quicksort.
\textit{Communications of ACM 4,7}
(1961),321-.
\bibitem{martin} 
\textsc{Martin, W. A.} Sorting.
\textit{ACM Computing Servey 3,4}(1971),147-174.
\bibitem{Rudra}
\textsc{Roy, S., Rudra, A., and Verma, A.} 
An energy complexity model for algorithms. In
\textit{Proceedings of the 4th Conference on Innovations in Theoretical Computer Science }(2013), ITCS '13,pp. 283-304
\end{thebibliography}



\end{document}